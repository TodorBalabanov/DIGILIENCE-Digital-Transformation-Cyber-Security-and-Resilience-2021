\documentclass[graybox]{svmult}

\usepackage{type1cm}
\usepackage{makeidx}
\usepackage{graphicx}
\usepackage{multicol}
\usepackage[bottom]{footmisc}
\usepackage{newtxtext}
\usepackage[varvw]{newtxmath}
\usepackage{hyperref}
\usepackage{amsmath}

\begin{document}

\title*{RSA Weaknesses Caused by RNG's Specifics}
\titlerunning{RSA Weaknesses Caused by RNG's Specifics}

\author{Ivan Blagoev, Todor Balabanov, Iliyan Iliev}
\authorrunning{I. Blagoev, T. Balabanov, I. Iliev}

\institute{Ivan Blagoev \at Institute of Information and Communication Technologies, Bulgarian Academy of Sciences, acad. Georgi Bonchev Str., block 2, 1113 Sofia, Bulgaria, \email{ivan.blagoev@iict.bas.bg}
\and Todor Balabanov \at Institute of Information and Communication Technologies, Bulgarian Academy of Sciences, acad. Georgi Bonchev Str., block 2, 1113 Sofia, Bulgaria \email{todor.balabanov@iict.bas.bg}
\and Iliyan Iliev \at Institute of Information and Communication Technologies, Bulgarian Academy of Sciences, acad. Georgi Bonchev Str., block 25, 1113 Sofia, Bulgaria \email{iliyan.iliev@iict.bas.bg}}

\maketitle

\abstract*{Rapid entry in digital transformation and Covid-19 have moved many activities to the Internet. The applying of cyber security tools gives a sense good security condition of the used digital services. This is often how things look on the surface, but the problems sometime is a hard to notice. The current study presents weaknesses in the still widely used cryptographic algorithm RSA, which allows RSA cryptographic keys to be compromise. As well as the connection with RNG as the root of all the resulting controversies around the issues under consideration.}

\abstract{Rapid entry in digital transformation and Covid-19 have moved many activities to the Internet. The applying of cyber security tools gives a sense good security condition of the used digital services. This is often how things look on the surface, but the problems sometime is a hard to notice. The current study presents weaknesses in the still widely used cryptographic algorithm RSA, which allows RSA cryptographic keys to be compromise. As well as the connection with RNG as the root of all the resulting controversies around the issues under consideration.}

\section{Introduction}
\label{sec:1}

The faster development of modern societies leads to greater digitalization. More activities and processes are much more productive and effectively managed by the involvement of the technologies. All these processes even did accelerate and did prove their value when the world was affected by the global pandemic of Covid-19. A transformation that would have taken years had to happen within months. The societies were pushed to search for different living styles much more connected with the technologies. At first glance, it looks that the world is prepared for such a technology challenge, and in general, it is like this. At the same time, the number of cybercrimes did rise. The encroachment on personal data, encroachment on personal money, and loss of information, extortion due to loss of information also escalated to unprecedented levels. All of this is a strong indicator that while technology and computing infrastructure have met the challenge, we are not ready for strong cybersecurity (Jang-Jaccard, 2014), (Kostadinov, 2019), (Dineva, 2019).  

Compliance with cybersecurity requirements is a prerequisite for the security and safety of IT infrastructures, digital resources, and personal data protection. In this respect, the topics of cryptography and the sufficiently reliable generation of random numbers that underlie any encryption system are of particular interest. (Velizar Shalamanov, 2020)

For modern cryptography needs, two types of random number generators are used - a true random number generator (TRNG) and a pseudo-random number generator (PRNG) (DiCarlo, 2012). 

True Random Number Generator (TRNG): applied when the RNG needs to generate values at a given time that should be unique and should not be repeated in subsequent RNG calls (Carr, 2003), (L'Ecuyer, 2007) . The numbers obtained with this type of RNG are applied to operations that require unique / non-repeating numerical values generated over time (Jin, 2004), (Camara, 2019). An example of such a situation is the generation of a cryptographic key for encoding/decoding data, initialization vectors, initial numerical values (seed) for controlled RNGs, etc. (Ergün, 2015; Ryabko, 2016). 

Pseudo-Random Number Generator (PRNG): An initial random number from the micro or macro world (seed) is used as the basis for this generator and a mathematical formula is used for subsequent numbers. From the initial value, by application of a particular algorithm, all random numbers generated subsequently originate. Subsequent values, in their order, are reproducible. The only unexpected and secret value that should be as unpredictable as possible is the initial number, which is the "root" at the base of this sequence and initiates the generation of the entire numeric sequence. From this technology are borrowed the authentication with One Time Password (OTP), the generation of cryptographic keys derived from the Master Root Key (applied in the compilation of portfolios in BlockChain - distributed ledger technology), authentication via HMAC, and others.

Traditional RNG security measures are mostly generalized statistics related to deviations from mathematical randomness (Trappe, 2006).

The hardware random number generator (HRNG) (Dichtl, 2003) or more True Random Number Generator (TRNG) is a device that generates random numbers from a physical process, not through an algorithm. This type of generators are radically different from those discussed so far, because such devices are often based on micro-world phenomena that generate low-level, statistically random "noise" signals, such as thermal noise, photoelectric effect including beam splitter and other quantum phenomena. These stochastic processes are considered completely unpredictable in theory, unlike the paradigm for generating pseudo-random numbers, often applied in computer programs. In general, two main sources of practical quantum-mechanical physical probabilities are known: quantum mechanics at the atomic or subatomic level and thermal noise (some of which are of quantum-mechanical origin). Quantum mechanics states that some physical phenomena, such as the nuclear decay of atoms, are fundamentally random and generally unpredictable.

Since the result of quantum mechanical events cannot be predicted, they are considered as a "gold standard" for generating random numbers. In fact, one of the best random number generators for server systems is considered to be photon-type quantum generators. They are compact enough and can fit on a PCB, while at the same time they have a very high-performance rate. According to some research, in many cases, such a hardware module has the capacity to power more than one public service server with quality random numbers. 

Regardless of which of the random number generators is applied (uncontrolled or controlled), the overall success of the system depends on the statistical qualities of the produced numbers (Lavasani, 2009). The rapidly growing demand for frequency bands, increasing volumes of stored data, and performing calculations, combined with the growing spectrum of cyber threats, ensure that our need for reliable and unpredictable random numbers will only grow in the future (Hart, 2017). 

The essence of RSA encryption is that it using only information that is publicly available. With the public key, anyone can encrypt a message they want to send to the owner of the private key. This is possible because without knowing the values of $p$ and $q$, no one but the owner of the private key can decode the message. Although everyone knows the public key $x = p * q$, this does not give them any effective way to find values for $p$ or $q$. According to a group of researchers years ago, it was thought that even the discovery of a 232-digit number would take more than 1,500 years of computational time (distributed among hundreds of computers) to compromise such a private key. 

On the surface, RSA encryption appears invulnerable. It could be said so far, but with the exception of one small problem, almost everyone uses the same random number generators. An excellent source of entropy is needed to generate the high-quality prime numbers that make up the cryptographic keys in RSA. In conventional computer systems, sources of quality entropy are relatively scarce for such a task. For this reason, seeds derived from quality entropy have been widely used for years. The calculations for the new RSA keys are then performed through pseudo-random number generators. 

Taking into account the facts, we can turn to a study of recent years, according to which a new idea is emerging, looking again at the well-known example: Let suppose that Bob and Alice publish public keys online. Because they both used the same program to generate random prime numbers, their public keys are more likely to have a common prime factor. Factoring Bob or Alice's public keys separately would be almost impossible but finding common factors between them is much easier. In fact, the time required to calculate the greatest common divisor between two numbers is close to proportional to the number of digits in the two numbers. Once the common divisor between Bob and Alice's keys has been identified, it can be invoiced to obtain the basic factorization of the two keys. From this point of view, it is possible to decode any messages sent to Bob or Alice.

Armed with this idea, researchers scanned the Internet and began collecting public keys from the algorithm. For this purpose, they collected 6.2 million real public keys. They then calculated the greatest common divisor between key pairs, compromising a key each time it shared a common factor with other keys. In this experiment, they managed to break 12,934 RSA keys. In other words, if the technology is used carelessly and the described weaknesses are not overcome, RSA encryption provides less than 99.8\% security. 

At first glance, this seems like the whole story. Reading the research on the subject (Ron was wrong, Whit is right) more closely reveals something more disturbing. According to the authors, they were able to perform the entire calculation in a few hours on a single CPU machine. Viewed through the theoretical foundation of RSA, it should be assumed that it will take years to calculate the GCD (greatest common divisor) between 36 trillion key pairs, rather than hours, according to the study.

How did they do it? The authors hint in a footnote that their calculation is based on an asymptotically fast algorithm that allows them to reduce the time to perform the calculations to almost linear. The actual description of the algorithm is kept secret by the reader, perhaps to prevent malicious use. Just a few months after the article was published, subsequent reports have already discussed in detail various approaches presenting fast algorithms (such as this study: Quasi-linear GCD computation and factoring RSA modules and even showing how to use GPUs to do the calculation with a rough force faster (Breaking weak 1024-bit RSA keys with CUDA). 

\url{https://digitalcommons.calpoly.edu/cgi/viewcontent.cgi?referer=https://scholar.google.com/&httpsredir=1&article=1241&context=csse_fac}

It can probably be said here that it is not good to mention things if they are to remain secret. On the other hand, if the weaknesses in cryptographic functions are not highlighted, we run the risk of being used by malicious individuals without the knowledge of others. In this case, in order to arrive at the results of the research, we must turn to the algorithms. 

Given the characteristics of cryptography and the proposed approach, the algorithm will deal with integers having an asymptotically large number of digits. Therefore, addition and multiplication will not be considered as fixed and relative time operations.

For $n$-bit numbers, take $O(n)$ time. Using a multiplication operation, multiplication seems to take $O(n^2)$ time. However, it turns out that there is an algorithm (Schönhage – Strassen algorithm) that works in time $O(n log 2 n log log n)$.

Calculating the GCD using the Euclidean algorithm takes $O(n2 log n log log n)$ time. Once again, however, researchers have found a better algorithm that works in time $O (n log 2 n log log n)$. Fortunately, all of these algorithms have already been implemented in GMP (GNU MP Subquadratic), the C ++ library for working with large numbers. For the rest of the study, we will use the $\overset{\sim}{\mathcal{O}}$ notation, a variant of the Big-O notation that ignores logarithmic factors. For example, while the calculation of GCD takes time $O (n log 2 n log log n)$, in notation we write that it takes time $\overset{\sim}{\mathcal{O}}(n)$.

\section{Problem Transformation}
\label{sec:2}

Define the set of public RSA keys with $k_1, ... , k_n$, where each key is the product of two large prime numbers. Note that $n$ is the total number of keys. Instead of calculating the GCD of each key pair, we can calculate for each key $k_i$ GCD of it and the product of all other keys $\prod_{t=1} K_t$. If the key $k_i$ shares one main factor with other keys, then this will give the main factor. However, if both main factors of $k_i$ are shared with other keys, the calculation will not be able to actually extract the individual primary factors. This case may be rare enough and not worth paying much attention to. 

Algorithm:

The algorithm has a slightly unusual recursive structure, as recursion occurs in the middle of the algorithm, not at the end.

At the beginning of the algorithm, all we have are the keys: $k_1, k_2, k_3, ...$

The first step of the algorithm is to connect the keys and calculate their results: $j_1=k_1K_2, j_2=k_3K_4, j_3=k5_K6, ...$

Then in recursion on the sequence of numbers $j_1, ..., j_n$ is calculated: $r_1 = GCD(j_1,\prod_{t<>1}j_t), r_2 = GCD(j_2,\prod_{t<>2}j_t), r_3 = GCD(j_3,\prod_{t<>3}j_t), ...$

The goal is to calculate $s_i = GCD (k_i, \prod_{t<>i} k_t)$ for each $k_j$ key. The important thing here is that when $i$ is odd, $s_i$ can be expressed as $s_i = GCD (k_i, r_{(i + 1)/2} k_{i+1})$ and that when $i$ is even, $s_i$ can be expressed as $s_i = GCD (k_i, r_{i/2} k_{i-1})$. 

To understand why this is so, one can check whether the expression on the right side of GCD is guaranteed to be a multiple of $s_i = GCD (k_i, \prod_{t<>i} k_t)$, while also being a divisor of $\prod_{t<>i} k_t$. This in turn suggests that the GCD calculation will be exactly $GCD(k_i, \prod_{t<>i} k_t)$, as expected.

Execution time:

Let $m$ denote the total number of bits needed to write $k_1, ..., k_n$. Each time the algorithm is repeated, it is ensured that the total number of bits in the recursion entry is not more than the previous recursion level. This is because new entries are products of pairs of elements from old ones.

Therefore, each of the levels of $O (log n)$ of recursion acts on input with a total size of $O (m)$ bits. In addition, the arithmetic operations in each recursion level take the most time $\overset{\sim}{\mathcal{O}} (m)$. Thus, the total operating time of the algorithm is also $\overset{\sim}{\mathcal{O}} (m)$ (since the recursion levels $O (log n)$ can be learned in the notation O-tilde).

If we expand the working time in standard Big-O notation, we get $O (m log3 m log log m)$. 

Is the approach practical?

At first glance, the triple logarithmic factor may seem to preclude the use of this algorithm. But in another study, it turns out that this presentation is quite reasonable. The article (Cloostermans, 2012) found that the algorithm takes approximately 7.65 seconds per thousand keys, which means that it will take just over 13 hours to execute 6.2 million keys.

It also turns out that one of the LOG factors can be eliminated using another approach that avoids GCD calculations altogether, except at the first level of recursion, for example, the article (Heninger, 2012). This improved algorithm takes about 4.5 seconds per thousand keys, resulting in a total run time of about 7.5 hours to work with 6.2 million keys. So, the calculation, which should take years, comes down to hours. All that is needed is the application of recursion, time series analysis to exploit the weakness in generating random numbers in systems.

In conclusion, it can be said that the weaknesses do not stem from an error in the arithmetic of RSA. They come from the technological weakness with which RSA is implemented. Computer systems, if they are of a newer generation, have hardware and software improvements that allow them to generate quality random numbers. However, the danger of this vulnerability remains. Because RSA needs really big random numbers. The current criteria for a reliable RSA key is a minimum of 2048 bits, and the recommended length is even 4096 bits. Other studies have also found that between 4096, 8192, and 16384 bits of an RSA key, the greater security of larger keys is minimal. The reason also comes from the limitations of random number generators. Larger RSA switches require extremely large real random numbers. Which is extremely difficult to obtain in a computer system. Even if the silicon module for HwRng is used for this purpose, the entropy buffer is 4096 bits and it accumulates slowly, with the limitations coming from the technology. When using RSA cryptography, in systems with significantly less hardware such as IoT, the generation of RSA keys will be even weaker. Again, the reasons are the same, and this type of device often does not have specialized hardware to enrich the entropy. For this reason, many such devices often become easy victims in cyber attacks.

As they say, at the heart of any encryption system is an algorithm and a generator for random numbers. Therefore, no matter how complex encryption algorithms are applied, they are considered to be as vulnerable as the random number generator that underlies this system.

The efficiency of RNG is measured by the degree of entropy to generate random numbers.

The complexity of analyzing a given random number generator is a function of the quality of its entropy, seasonality, and tendency to collide. These are the moments when the random number generator will generate a value that is a cyclic or value field, which leads to the repetition or generation of a new but expected value. Through time series mathematics, it is possible to determine the entropy over time and it is likely to calculate (predict) the possible future reappearance of the data. The detection of seasonality in the obtained values, deviations, or collisions may also indicate weaknesses of the random number generator. If the generator is of good quality, then it will follow the analysis of a very large number of statistical values from a numerical array generated by it with a high degree of entropy and unpredictability, which will be very resource-intensive and complex. This will also make it very attack-resistant throughout the cryptography associated with this generator.

\section{Conclusions}
\label{sec:3}

The study of the presented weaknesses in the asymmetric RSA algorithm is of high importance for a more secure and fast transition to the modern digital transformation. Hardware solutions that could significantly support the quality of RNG in computer systems have been listed. Which would significantly affect the security of modular cryptography. The study also reveals a RSA's cryptographic security problems when using it. Which would lead to search for replacement of modular cryptography with other solutions that are more secure when using the widespread in computer systems RNG solutions.

\begin{acknowledgement}
This research is partially supported by the Bulgarian Ministry of Education and Science (contract D01–205/23.11.2018) under the National Scientific Program ``Information and Communication Technologies for a Single Digital Market in Science, Education and Security (ICTinSES)'', approved by DCM \# 577/17.08.2018.
\end{acknowledgement}

\begin{thebibliography}{99.}

\bibitem{01} Arjen K. Lenstra, James P. Hughes, Maxime Augier, Joppe W. Bos, Thorsten Kleinjung, Christophe Wachter: Ron was wrong, Whit is right, https://eprint.iacr.org/2012/064.pdf (2012)

\bibitem{02} Kerry Scharfglass, Darrin Weng, Joseph White, Christopher Lupo, Breaking weak 1024-bit RSA keys with CUDA, https://digitalcommons.calpoly.edu/cgi/viewcontent.cgi? referer=https://scholar.google.com/\&httpsredir=1\&article=1241\&context=csse\_fac

\bibitem{03} Cloostermans B., Quasi-linear GCD computation and factoring RSA moduli, Eindhoven University of Technology, Department of Mathematics and Computer Science, Bachelor Mathematics (2012)

\bibitem{04} Heninger N., Durumeric Z., Wustrow E., Halderman J. A., Mining Your Ps and Qs: Detection of Widespread Weak Keys in Network Devices, 21 st Security Symposium Security’12, ISBN 978-931971-95-9, Bellevue, WA, pp. 205—220 (2012)

\bibitem{18} Arjen K. Lenstra, James P. Hughes, Maxime Augier, Joppe W. Bos, Thorsten Kleinjung, and Christophe Wachter, “Ron was wrong, Whit is right”, I\&C - School of Computer and Communication Sciences - LACAL - Laboratory for cryptologic algorithms, https://eprint.iacr.org/2012/064.pdf, (2012).

\bibitem{19} Kerry Scharfglass, Darrin Weng, Joseph White, Christopher Lupo, "Breaking weak 1024-bit RSA keys with CUDA", California Polytechnic State University, Dec 2012, Publisher: IEEE, DOI: 10.1109/PDCAT.2012.58ISBN: 978-0-7695-4879-1

\bibitem{05} Li C., Zhang J., Sang L., Gong L., Wang L., Wang A., Wang Y., Deep Learning- Based Security Verification for a Random Number Generator Using White Chaos, Entropy, 22, 1134; doi:10.3390/e22101134 (2020)

\bibitem{06} Singh S., Maakar S. K. and Kumar S., A Performance Analysis of DES and RSA Cryptography, International Journal of Emerging Trends \& Technology in Computer Science (IJETTCS), Volume 2, Issue 3, ISSN 2278-6856 (2013)

\bibitem{07} Velizar Shalamanov, Vladimir Monov, Ivaylo Blagoev, Silvia Matern, Gergana Vassileva, Ivan Blagoev, A Model of ICT Competence Development for Digital Transformation. ISSN 0861-5160 (print), ISSN 1314-2119 (online) (2020)

\bibitem{20} Julian Jang-Jaccard and SuryaNepal, “A survey of emerging threats in cybersecurity,” Journal of Computer and System Sciences 80, no. 5 (2014): 973-993. 

\bibitem{08} Jang-Jaccard J., Nepal S., A survey of emerging threats in cybersecurity. Journal of Computer and System Sciences, 80(5), 973-993 (2014)

\bibitem{09} Kostadinov G., Atanasova T., Security Policies for Wireless and Network Infrastructure. Problems of Engineering Cybernetics and Robotics, vol. 71, 14-19, Bulgarian Academy of Sciences (2019)

\bibitem{10} Dineva, K., Atanasova, T.: Regression Analysis on Data Received from Modular IoT System. ESM’2019, EUROSIS-ETI, ISBN: 978-9492859-09-9, EAN: 9789492859099, pp.114-118, 2019

\bibitem{11} DiCarlo D., Random Number Generation: Types and Techniques, Liberty University, (2012)

\bibitem{12} Jin, A., Ling, D., Goh A., Biohashing: Two factor authentication featuring fingerprint data and tokenised random number. Pattern Recognition, 37, 2245- 2255. (2004)

\bibitem{13} Ergün S., Security analysis of a chaos-based random number generator for applications in cryptography, 15th International Symposium on Communications and Information Technologies (ISCIT), pp. 319-322, doi: 10.1109/ISCIT.2015.7458371 (2015)

\bibitem{14} Ryabko B., Astola J., Malyutov M., Compression-Based Methods of Statistical Analysis and Prediction of Time Series, Springer International Publishing Switzerland, eBook ISBN 978-3-319-32253-7 (2016)

\bibitem{15} Trappe, L., Washington, L. Introduction to cryptography with coding theory (2nd ed). Upper Saddle River, NJ: Pearson (2006)

\bibitem{16} Hart J. D., Roy R. and Murphy T. E., Optical random number generation - harvesting entropy from noise and chaos, 51st Annual Conference on Information Sciences and Systems (CISS), doi: 10.1109/CISS.2017.7926165 (2017)

\bibitem{17} Lavasani, A., Eghlidos, T. Practical next bit test for evaluating pseudorandom sequences. Scientia Iranica, 16(1), 19-33 (2009)

\bibitem{21} Dichtl, M. How to predict the output of a hardware random number generator. CHES 2003, 2779, 181-188. (2003) Dichtl, M. How to predict the output of a hardware random number generator. CHES 2003, 2779, 181-188. (2003).

\bibitem{22} Carr J., Simple random number generation, Computers \& Geosciences, 29(10):1269-1275 (2003).

\bibitem{23} L’Ecuyer P., Random Number Generation, In book: Handbook of Simulation: Principles, Methodology, Advances, Applications, and Practice (2007)

\bibitem{24} Camara C., Martín H., Peris-Lopez P., Aldalaien M., Design and Analysis of a True Random Number Generator Based on GSR Signals for Body Sensor Networks, Sensors 19, 2033; doi:10.3390/s19092033 (2019)

\end{thebibliography}


\end{document}
